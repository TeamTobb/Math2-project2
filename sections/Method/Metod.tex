I denne delen skal du beskrive nøyaktig hvordan du har gjort arbeidet og hvorfor du har valgt a gjøre det pa denne maten. Hvis du gjør en utviklingsoppgave skal du beskrive arbeidsprosessen din, valgene du har tatt underveis og hvorfor du har gjort det du har gjort. Grunnen til at du skal gjøre dette, er at dette pa mange mater er ditt bevis pa dine resultater. Nar du skal beskrive resultatene dine ma disse kunne etterprøves, og dette kan kun gjøres hvis du beskriver nøyaktig hva du har gjort og hvordan. En rapport med en svak metodedel vil i verste fall kunne bli underkjent. Det er ogsa viktig at du begrunner alle valg du har tatt underveis i arbeidet ditt. Hvis du for eksempel skriver en state-of-the-art oppgave, er det viktig at du begrunner hvorfor du har valgt akkurat de informasjonskildene som du har benyttet, og hvorfor andre kilder til informasjon har blitt ekskludert fra ditt utvalg.