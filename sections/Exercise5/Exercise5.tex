% !TEX encoding = UTF-8 Unicode
% \documentclass{article}
% \usepackage{../../superstyle}
% \usepackage{listings}
% \usepackage{amsmath}
% \begin{document}
% remove all before

%oppgavetekst
Now repeat Step 4 with a more tightly grouped set of satellites. Choose all $\phi$$_i$ within
5 percent of one another and all $\theta$$_i$ within 5 percent of one another. Solve with and without the same input error as in Step 4. Find the maximum position error and error magnification factor. Compare the conditioning of the GPS problem when the satellites are tightly or loosely bunched.

\vspace{5mm}

\textbf{Løsning}

\begin{lstlisting}[caption={Task3.m}]
function [conditionNumber, worst_max_pos_error] = task5()

rho = 26570;

phi = [pi/4 (pi/4)*1.015 (pi/4)*1.03 (pi/4)*1.045];
theta = [pi pi*1.015 pi*1.030 pi*1.045];
A = ones(1,4); 
B = ones(1,4); 
C = ones(1,4);

for i=1:4
    A(i) = rho * cos(phi(i)) * cos(theta(i));
    B(i) = rho * cos(phi(i)) * sin(theta(i));
    C(i) = rho * sin(phi(i));
end

c = 299792.458;

% finner satelittavstandene R og tilhørende reisetider for signalene.
R = sqrt((A.^2)+(B.^2)+((C-6370).^2));
t = 0.0001 + (R/c);

[x,y,z,d]= task2general(A,B,C,t); 
riktig = [x y z d];

% lager en matrise med alle mulige kombinasjoner av + og - (2^4).
factor = [[-1,-1,-1,-1];[1,1,1,1];[-1,-1,-1,1];[-1,-1,1,1];
		  [-1,1,1,1];[-1,1,-1,1];[-1,1,1,-1];[-1,-1,1,-1];
		  [-1,1,-1,-1];[1,1,1,-1];[1,1,-1,-1];[1,-1,-1,-1];
		  [1,-1,1,-1];[1,-1,-1,1];[1,1,-1,1];[1,-1,1,1]];

max_pos_error = zeros(1,16);
emf = zeros(1,16);

% finner alle feil i posisjon og EMFer for alle kombinasjoner av + og -
for i = 1:length(factor)
    dt1(1) = t(1) + 1e-8*factor(i,1);
    dt1(2) = t(2) + 1e-8*factor(i,2);
    dt1(3) = t(3) + 1e-8*factor(i,3);
    dt1(4) = t(4) + 1e-8*factor(i,4); 
    [x2,y2,z2,d2]=task2general(A,B,C,dt1);
    feil = [x2 y2 z2 d2];
    max_pos_error(i) = max(abs(riktig-feil));
    emf(i) = max_pos_error(i)/(c*max(abs(dt1-t)));
end

worst_max_pos_error = max(max_pos_error);
conditionNumber = max(emf);

end
\end{lstlisting}

\begin{figure}[h]
    \centering
    \includegraphics[width=0.9\textwidth]{sections/Exercise5/task5result}
    % \includegraphics[width=0.9\textwidth]{task5result}
    \caption{Matlabresultat fra oppgave 5}
    \label{fig:task5result}
\end{figure}

Som man ser i figur \ref{fig:task5result} er både kondisjonstallet og maksimum posisjonsfeil $\approx10^5$ høyere nå som alle satelittene er satt under 5\% fra hverandre i distanse kontra avstandene i exercise 4.

%tabeller
	\begin{figure}[h]
		\begin{minipage}{.5\textwidth}
			\centering
			\includegraphics[width=0.8\textwidth]{sections/Exercise5/emf.png}
			% \includegraphics[width=0.8\textwidth]{emf.png}
				\caption{Task 5 - EMF tabell}
				\label{fig:task5EMF}
		\end{minipage}
		\vspace{20 mm}
		\begin{minipage}{.5\textwidth}
			Grafen i figur \ref{fig:task5EMF_graph} har en logaritmisk y-akse og man kan se ut den og av figur \ref{fig:task5EMF} at det er stor forskjell ($\approx10^9$) fra når man bruker samme input-error som i exercise 4 kontra uten([-,-,-,-] og [+,+,+,+]).
		\end{minipage}
		%\vspace{20 mm}

		\centering
	    \includegraphics[width=0.8\textwidth]{sections/Exercise5/emf_graph.png}
	    % \includegraphics[width=0.8\textwidth]{emf_graph.png}
		    \caption{Task 5 - EMF graf}
		    \label{fig:task5EMF_graph}
	\end{figure}

	\begin{figure}
		\begin{minipage}{.5\textwidth}
			\centering
			\includegraphics[width=0.8\textwidth]{sections/Exercise5/max_pos_error.png}
			% \includegraphics[width=0.8\textwidth]{max_pos_error.png}
		    	\caption{Task 5 - Maks. pos. tabell}
		    	\label{fig:task5max_pos_error}
		\end{minipage}
		\vspace{20 mm}
		\begin{minipage}{.5\textwidth}
			På samme måte som i figur \ref{fig:task5EMF_graph} har figur \ref{fig:task5max_pos_error} en logaritmisk y-akse. Det er også likhet i forskjellen når man bruker samme input-error som i exercise 4 og ikke.
		\end{minipage}

		\centering
		\includegraphics[width=0.8\textwidth]{sections/Exercise5/max_pos_error_graph.png}
		% \includegraphics[width=0.8\textwidth]{max_pos_error_graph.png}
		    \caption{Task 5 - Maksimal posisjonsfeil graf}
		    \label{fig:task5max_pos_error_graph}
	\end{figure}
%end tabeller

% remove after
% \end{document}