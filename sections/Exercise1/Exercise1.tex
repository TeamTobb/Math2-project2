%oppgavetekst
I oppgaven får vi gitt systemet: 
\begin{align}
{(x - {A_1})^2} + {(y - {B_1})^2} + {(z - {C_1})^2} &= {[c({t_1} - d)]^2} \nonumber \\ 
{(x - {A_2})^2} + {(y - {B_2})^2} + {(z - {C_2})^2} &= {[c({t_2} - d)]^2}  \nonumber \\
{(x - {A_3})^2} + {(y - {B_3})^2} + {(z - {C_3})^2} &= {[c({t_3} - d)]^2}  \nonumber \\
{(x - {A_4})^2} + {(y - {B_4})^2} + {(z - {C_4})^2} &= {[c({t_4} - d)]^2} \label{Ex1:system}
\end{align}

Oppgaveteksten sier at hvis man subtraherer de tre siste likningene med den første, får vi tre linneære likninger. Vi starter med å subtrahere den andre likningen med den første i (\ref{Ex1:system}). 
\begin{multline}
{(x - {A_1})^2} - {(x - {A_2})^2} + {(y - {B_1})^2} - {(y - {B_2})^2} + {(z - {C_1})^2} - {(z - {C_2})^2} = 
{[c({t_1} - d)]^2} - {[c({t_2} - d)]^2}  \label{Ex1:linearEqs}
\end{multline}


Vi løser opp andregradsuttrykkene i (\ref{Ex1:linearEqs}) og setter uttrykket lik 0.
\begin{multline}
\\
({x^2} - 2{A_1}x + A_1^2) - ({x^2} - 2{A_2}x + A_2^2)  \\
+ \\
({y^2} - 2{B_1}y + B_1^2) - ({y^2} - 2{B_2}y + B_2^2) \\
+ \\
({z^2} - 2{C_1}z + C_1^2) - ({z^2} - 2{C_2}z + C_2^2)\\
- \\
({c^2}{t_1} - 2{t_1}{c^2} + {c^2}{d^2}) + ({c^2}{t_2} - 2{t_2}{c^2} + {c^2}{d^2}) \\
= \\
0 \\ \nonumber
\end{multline} 


Vi samler så utrykkene for henholdsvis A, B, C og de resterende konstantene uttrykt med $W = {A_1}^2 - {A_2}^2 + {B_1}^2 - {B_2}^2 + {C_1}^2 - {C_2}^2 + ({c^2}( - {t_1}^2 + {t_2}^2))$.  
\begin{multline}
- 2x({A_1} - {A_2}) - 2y({B_1} - {B_2}) - 2z({C_1} - {C_2}) + 2d{c^2}({t_1} - {t_2}) + W = 0
\end{multline}


Det blir flere lignende likninger når vi trekker fra likning nr 3 og 4 fra den første i (\ref{Ex1:system}). De tre likningene blir
\begin{align}
- 2x({A_1} - {A_2}) - 2y({B_1} - {B_2}) - 2z({C_1} - {C_2}) + 2d{c^2}({t_1} - {t_2}) + W &= 0 \nonumber \\
- 2x({A_1} - {A_3}) - 2y({B_1} - {B_3}) - 2z({C_1} - {C_3}) + 2d{c^2}({t_1} - {t_3}) + W &= 0  \nonumber\\
- 2x({A_1} - {A_4}) - 2y({B_1} - {B_4}) - 2z({C_1} - {C_4}) + 2d{c^2}({t_1} - {t_4}) + W &= 0 \nonumber \\ \label{Ex1:3equations}
\end{align}


Likningene blir så samlet slik at x-, y-, z-, og d-punktene blir samlet i hver sin vektor. Vektorene uttrykkes ved ${\vec u_x}, {\vec u_y}, {\vec u_z}$ og  ${\vec u_d}$, samt $\vec{w}$. 
\begin{align}
{\vec u_x} &=  - 2*[{A_1} - {A_2};\enspace {A_1} - {A_3}; \enspace{A_1} - {A_4}] \nonumber \\
{\vec u_y} &=  - 2*[{B_1} - {B_2}; \enspace{B_1} - {B_3}; \enspace{B_1} - {B_4}] \nonumber \\
{\vec u_z} &=  - 2*[{C_1} - {C_2}; \enspace{C_1} - {C_3}; \enspace{C_1} - {C_4}] \nonumber \\
{\vec u_d} &=  2c^2*[{t_1} - {t_2}; \enspace{t_1} - {t_3}; \enspace{t_1} - {t_4}] \nonumber \\
{\vec w}  &= [{A_1}^2 - {A_2}^2 + {B_1}^2 - {B_2}^2 + {C_1}^2 - {C_2}^2 + ({c^2}( - {t_1}^2 + {t_2}^2)); \nonumber \\
&\hskip 1.4em  {A_1}^2 - {A_3}^2 + {B_1}^2 - {B_3}^2 + {C_1}^2 - {C_3}^2 + ({c^2}( - {t_1}^2 + {t_3}^2)); \nonumber \\ 
&\hskip 1.4em {A_1}^2 - {A_4}^2 + {B_1}^2 - {B_4}^2 + {C_1}^2 - {C_4}^2 + ({c^2}( - {t_1}^2 + {t_4}^2));]
\end{align}
Vi får da uttrykket $x*{\vec u_x} + y*{\vec u_y} + z*{\vec u_z} + d*{\vec u_d} + {\vec w} = 0$. 

%jorgen kommer her 

I boken står det at vi kan komme frem til en formel for x, uttrykt ved d, fra uttrykket $0=\text{det}[\vec{u}_y | \vec{u}_z | x\vec{u}_x + y\vec{u}_y + z\vec{u}_z + d\vec{u}_d + w\vec{u}_w]$. Vi setter opp matrisen: 

\begin{align} \label{eq:det}
	\begin{vmatrix}
	&\vec{u}_y\\
	&\vec{u}_z \\
	&x\vec{u}_x + y\vec{u}_y + z\vec{u}_z + d\vec{u}_d + W
	\end{vmatrix}
	=0
\end{align}

Vi ønsker å utlede et uttrykk for x fra denne determinanten. Determinanten for matrisen i likning \ref{eq:det} kan skrives om på følgende måte: 

\begin{align} \label{eq:det_divided}
	\begin{vmatrix}
	&\vec{u}_y\\
	&\vec{u}_z \\
	&x\vec{u}_x
	\end{vmatrix}
	+
	\begin{vmatrix}
	&\vec{u}_y\\
	&\vec{u}_z \\
	&y\vec{u}_y
	\end{vmatrix}
	+
	\begin{vmatrix}
	&\vec{u}_y\\
	&\vec{u}_z \\
	&z\vec{u}_z
	\end{vmatrix}
	+
	\begin{vmatrix}
	&\vec{u}_y\\
	&\vec{u}_z \\
	&d\vec{u}_d
	\end{vmatrix}
	+
	\begin{vmatrix}
	&\vec{u}_y\\
	&\vec{u}_z \\
	&W
	\end{vmatrix}
	=0
\end{align}

Videre bruker vi loven som sier at matriser med repeterende rader eller kolonner er 0. Da står vi igjen med: 

\begin{align} \label{eq:det_final}
	\begin{vmatrix}
	&\vec{u}_y\\
	&\vec{u}_z \\
	&x\vec{u}_x
	\end{vmatrix}
	+
	\begin{vmatrix}
	&\vec{u}_y\\
	&\vec{u}_z \\
	&d\vec{u}_d
	\end{vmatrix}
	+
	\begin{vmatrix}
	&\vec{u}_y\\
	&\vec{u}_z \\
	&W
	\end{vmatrix}
	=0
\end{align}

Den første delen av determinanten uttrykt i ligning \ref{eq:det_final} kan løses opp og få x på utsiden. Det samme kan gjøres hvor d inngår i matrisen. Da står vi igjen med: 

\begin{align}
&x\cdot \begin{vmatrix}
	&\vec{u}_y\\
	&\vec{u}_z \\
	&\vec{u}_x
	\end{vmatrix}
	+
	d\cdot
	\begin{vmatrix}
	&\vec{u}_y\\
	&\vec{u}_z \\
	&\vec{u}_d
	\end{vmatrix}
	+
	\begin{vmatrix}
	&\vec{u}_y\\
	&\vec{u}_z \\
	&W
	\end{vmatrix}
	=0 \nonumber \\ 
	&x\cdot \begin{vmatrix}
	&\vec{u}_y\\
	&\vec{u}_z \\
	&\vec{u}_x
	\end{vmatrix}
	=-d\cdot
	\begin{vmatrix}
	&\vec{u}_y\\
	&\vec{u}_z \\
	&\vec{u}_d
	\end{vmatrix}
	-
	\begin{vmatrix}
	&\vec{u}_y\\
	&\vec{u}_z \\
	&W
	\end{vmatrix} \nonumber 
\end{align}
Til slutt bruker vi loven om at en determinant er den samme uavhengig av fortegn. $-1/cdot$ en determinant er fortsatt det samme tallet. Da står vi igjen med: 

\begin{align}
    x=\frac{d\cdot
    \text{det}
	\begin{bmatrix}
	&\vec{u}_y\\
	&\vec{u}_z \\
	&\vec{u}_d
	\end{bmatrix}+\text{det}\begin{bmatrix}
	&\vec{u}_y\\
	&\vec{u}_z \\
	&W
	\end{bmatrix}}{\text{det}\begin{bmatrix}
	&\vec{u}_y\\
	&\vec{u}_z \\
	&\vec{u}_x
	\end{bmatrix}}
\end{align}

For å finne uttrykk for henholdsvis y og z bruker vi samme metode, men med forskjellige utgangsmatriser. For y har vi følgende matrise som start: 

\begin{align}
    \begin{vmatrix}
	&\vec{u}_x\\
	&\vec{u}_z \\
	&x\vec{u}_x + y\vec{u}_y + z\vec{u}_z + d\vec{u}_d + W
	\end{vmatrix}
	=0
\end{align}
For z har vi følgende matrise som start: 
\begin{align}
    \begin{vmatrix}
	&\vec{u}_x\\
	&\vec{u}_y \\
	&x\vec{u}_x + y\vec{u}_y + z\vec{u}_z + d\vec{u}_d + W
	\end{vmatrix}
	=0
\end{align}

Bruker vi samme fremgangsmåte som for x, ender vi opp med følgende uttrykk for x, y og z: 

\begin{align}
     x&=\frac{d\cdot
    \text{det}
	\begin{bmatrix}
	&\vec{u}_y\\
	&\vec{u}_z \\
	&\vec{u}_d
	\end{bmatrix}+\text{det}\begin{bmatrix}
	&\vec{u}_y\\
	&\vec{u}_z \\
	&W
	\end{bmatrix}}{\text{det}\begin{bmatrix}
	&\vec{u}_y\\
	&\vec{u}_z \\
	&\vec{u}_x
	\end{bmatrix}} 
\end{align}

\begin{align}
    y&=\frac{d\cdot
    \text{det}
	\begin{bmatrix}
	&\vec{u}_x\\
	&\vec{u}_z \\
	&\vec{u}_d
	\end{bmatrix}+\text{det}\begin{bmatrix}
	&\vec{u}_x\\
	&\vec{u}_z \\
	&W
	\end{bmatrix}}{\text{det}\begin{bmatrix}
	&\vec{u}_y\\
	&\vec{u}_z \\
	&\vec{u}_x
	\end{bmatrix}}
\end{align}
\begin{align}
    z&=\frac{d\cdot
    \text{det}
	\begin{bmatrix}
	&\vec{u}_x\\
	&\vec{u}_y \\
	&\vec{u}_d
	\end{bmatrix}+\text{det}\begin{bmatrix}
	&\vec{u}_x\\
	&\vec{u}_y \\
	&W
	\end{bmatrix}}{\text{det}\begin{bmatrix}
	&\vec{u}_y\\
	&\vec{u}_z \\
	&\vec{u}_x
	\end{bmatrix}}
\end{align}

Nå har vi altså fått uttrykt både x, y og z ved d. For nå å gå videre, setter vi disse inn i bokas likning 4.38. Dette vil resultere i en likning som inneholder d i første og andre grad, og vi kan derav finne d ved å bruke andregradsformelen. Med adnre ord vil vi få ett utrykk som vi kan skrive som $Ad^2+Bd+C=0$ Andregradsformelen er gitt ved $d=\frac{-B \pm \sqrt{B^2-4AC}}{2A}$. Før vi setter uttrykkene vi har funnet for x, y og z inn i bokas likning 4.38, tilegner vi de variabler for å gjøre uttrykket ryddigere. Vi setter $x_1=\frac{d\cdot
    \text{det}
	\begin{bmatrix}
	&\vec{u}_y\\
	&\vec{u}_z \\
	&\vec{u}_d
	\end{bmatrix}}{\text{det}\begin{bmatrix}
	&\vec{u}_y\\
	&\vec{u}_z \\
	&\vec{u}_x
	\end{bmatrix}}$, $x_2=\frac{d\cdot
    \text{det}
	\begin{bmatrix}
	&\vec{u}_y\\
	&\vec{u}_z \\
	&\vec{u}_w
	\end{bmatrix}}{\text{det}\begin{bmatrix}
	&\vec{u}_y\\
	&\vec{u}_z \\
	&\vec{u}_x
	\end{bmatrix}}$, $y_1=\frac{d\cdot
    \text{det}
	\begin{bmatrix}
	&\vec{u}_x\\
	&\vec{u}_z \\
	&\vec{u}_d
	\end{bmatrix}}{\text{det}\begin{bmatrix}
	&\vec{u}_y\\
	&\vec{u}_z \\
	&\vec{u}_x
	\end{bmatrix}}$, $y_2=\frac{d\cdot
    \text{det}
	\begin{bmatrix}
	&\vec{u}_x\\
	&\vec{u}_z \\
	&\vec{u}_w
	\end{bmatrix}}{\text{det}\begin{bmatrix}
	&\vec{u}_y\\
	&\vec{u}_z \\
	&\vec{u}_x
	\end{bmatrix}}$, $z_1=\frac{d\cdot
    \text{det}
	\begin{bmatrix}
	&\vec{u}_x\\
	&\vec{u}_y \\
	&\vec{u}_d
	\end{bmatrix}}{\text{det}\begin{bmatrix}
	&\vec{u}_y\\
	&\vec{u}_z \\
	&\vec{u}_x
	\end{bmatrix}}$, $z_2=\frac{d\cdot
    \text{det}
	\begin{bmatrix}
	&\vec{u}_x\\
	&\vec{u}_y \\
	&\vec{u}_w
	\end{bmatrix}}{\text{det}\begin{bmatrix}
	&\vec{u}_y\\
	&\vec{u}_z \\
	&\vec{u}_x
	\end{bmatrix}}$.

Av dette følger det at $x=x_1+x_2$, $y=y_1+y_2$ og $z=z_1+z_2$. Setter vi nå dette inn i første linkningen i bokas likning 4.38 får vi: 

\begin{multline}
	\\
    ((x_1+x_2)-A_1)^2+((y_1+y_2)-B_1)^2+((z_1+z_2)-C_1)^2\\
    =\\
    (c(t_1-d))^2 \\
\end{multline}
\begin{multline}
	\\
  (x_1+x_2)^2-2A(x_1+x_2)+A^2+(y_1+y_2)^2-2B(y_1+y_2)+B^2\\
    +\\
    (z_1+z_2)^2-2C(z_1+z_2)+C^2-(c(t_1-d))^2\\
    =0 \nonumber 	\\
\end{multline}

\begin{multline}\label{eq:d_expanded}
    \\
    x_1^2+2x_1x_2+x_2^2-2Ax_1-2Ax_2+A^2+y_1^2+2y_1y_2+y_2^2-2By_1-2By_2+B^2\\
    +\\
    z_1^2+2z_1z_2+z_2^2-2Cz_1-2Cz_2+C^2-c^2t^2-c^22td-c^2d^2 \\
    = 0\\
\end{multline}
Som vi ser i figur \ref{eq:d_expanded} har vi satt inn uttrykkene vi har funnet for x, y og z. Videre ønsker vi å samle alle leddene som vil få $d^2$ for seg, leddene med $d^1$ for seg og resten i en siste gruppe. Da vi ga verdier til $x_1$, $y_1$ og $z_1$, ser vi at det er bare disse leddene som inneholder en d. Det vil si at de eneste leddene hvor d blir i andre grad, er leddene hvor $x_1$, $y_1$ og $z_1$ er opphøyd i andre. Disse samler vi og setter inn i A i andregradsformelen. Videre vil alle leddene som inneholder d i første grad være der hvor $x_1$, $y_1$ og $z_1$ står i første grad. Disse samler vi i B i andregradsformelen. Vi har altså samlet alle koeffisientene som tår foran $d^2$ i A, koeffisientene foran $d^1$ i B mens resten av leddene samles i C. A, B og C blir da som følger: 

\begin{align}
    A&=x_1^2+y_1^2+z_1^2-c^2 \\
    B&=2x_1x_2-2Ax_1+2y_1y_2-2By_1+2z_1z_2-2Cz_1-c^22t\\
    C&=x_2^2-2Ax_2+A^2+y_2^2-2By_2+B^2+z_2^2-2Cz_2+C^2-c^2t^2
\end{align}
Dette setter vi inn i andregradsformelen, og får derav 2 verdier for d. 

\begin{align}
    d=\frac{-(2x_1x_2-2Ax_1+2y_1y_2-2By_1+2z_1z_2-2Cz_1-c^22t) \pm \sqrt{(2x_1x_2-2Ax_1+2y_1y_2-2By_1+2z_1z_2-2Cz_1-c^22t)^2-4(x_1^2+y_1^2+z_1^2-c^2 )\cdot(x_2^2-2Ax_2+A^2+y_2^2-2By_2+B^2+z_2^2-2Cz_2+C^2-c^2t^2)}}{2A}
\end{align}
\end{document}
