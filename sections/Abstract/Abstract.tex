

Denne rapporten omhandler Euler-Bernoulli bjelken (The Euler-Bernoulli beam). Dette er en fundamental modell som sier hvordan forskjellige materialer bøyer seg under påvirkning av diverse krefter. Modellens nøyaktighet ble praktisk demonstrert da den ble brukt til å bygge både Eiffel-tårnet og Pariserhjulet \cite{NdogmoEB}.  Det viser seg at diskretisering av differentialligningen gjør at vi får et system av lineære ligninger som kan brukes til å finne den vertikale forskyvningen. Jo mindre steglengde vi bruker til diskretiseringen, jo større blir likningssystemet. Euler-Bernoulli modellen er en meget sentral modell innenfor dette området, og det er skrevet mange artikler rundt dette. For eksempel har Timoshenko foreslått en videre forbedring av bjelketeorien\cite{Timoshenko} fordi Euler-Bernoulli modellen passer best til smalere bjelker med liten vibrasjon.\\

I rapporten går vi først igjennom grunnleggende teori som utgjør grunnlaget for utregningene gjort for å regne ut den vertikale forskyvningen i hvert punkt langs bjelken. På grunn av at oppgaven vi har fått er en faktisk oppgave som står i læreboken, går vi i teoridelen igjennom de fleste formlene boken oppgir, med noen få unntak som spesifikt ikke skulle beskrives. Til slutt i teoridelen følger det 2 bevis. Det ene beviset beviser likning 2.28, som er et uttrykk på den fjerdederiverte til funksjonen y i Euler-Bernoulli likningen. Av grunner beskrevet i teoridelen er ikke denne likningen brukbar langs hele planken, og derfor følger et nytt bevis på den fjerdederiverte i punktet $x_1$ etter dette. \\

Etter teoridelen løser vi en rekke oppgaver. Oppgavene vi løser er oppgavene som står til slutt i Reality Check 2, og går ut på forskjellige utregninger knyttet til et stupebrett, som enkelt kan relateres til Euler-Bernoulli-likningen. Oppgavene går fra å lage diverse MatLab-programmer som regner ut den vertikale forskyvningen langs planken, på forskjellige måter, til både det å finne hvor store feil forskjellige funksjoner har. Vi får også oppgitt en løsning på funksjonen, som vi skal vise at tilfredsstiller Euler-Bernoulli funksjonen, gitt en viss påført kraft. Til slutt regner vi også den vertikale forskyvningen når vi påfører vekt på stupebrettet. \\ 