
Denne rapporten tar for seg GPS-problemet (Global Positioning System). GPS er et system bestående av en rekke satelitter som ved hjelp av avstander og tidsforskjeller kan regne ut posisjonen til brukere over hele jordkloden. Da GPS først ble tatt ibruk revolusjonerte det hvordan mennesker generelt kunne lokalisere seg. Idag er GPS er et system som er i ekstremt mye bruk, og man finner det implementert i for eksempel smarttelefoner, bærbare datamaskiner og klokker. Ved å innføre en fjerde satelitts posisjon, kan vi eliminere det originale tidsproblemet for å finne en meget nøyaktig posisjon, i tillegg til den synkroniserte tiden til satelittklokkene. \\

I rapporten går vi først igjennom grunnleggende teori som ligger til utregningene gjort i oppgavene. Vi tar utgangspunkt i formlene boka oppgir, og løser GPS-problemet på bakgrunn av disse. \\

Etter teoridelen løser vi en rekke oppgaver knyttet til GPS-problemet og dets nøyaktighet. Oppgavene går fra å finne forskjellige løsninger på det originale likningssystemet, implementere disse i MatLab, i tillegg til å regne og utføre eksperimenter på hvor nøyaktige resultatene er. Vi viser også hvordan satelittenes posisjon har en påvirkning på det endelige resultatet. 